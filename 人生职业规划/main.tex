\documentclass{article}
\usepackage[UTF8]{ctex}
\usepackage{geometry}
\usepackage{multirow}
\usepackage{natbib}
\geometry{left=3.18cm,right=3.18cm,top=2.54cm,bottom=2.54cm}
\usepackage{graphicx}
\pagestyle{plain}	
\usepackage{setspace}
\usepackage{enumerate}
\usepackage{caption2}
\usepackage{datetime} %日期
\renewcommand{\today}{\number\year 年 \number\month 月 \number\day 日}
\renewcommand{\captionlabelfont}{\small}
\renewcommand{\captionfont}{\small}
\begin{document}

\begin{figure}
    \centering
    \includegraphics[width=8cm]{upc.png}

    \label{figupc}
\end{figure}

	\begin{center}
		\quad \\
		\quad \\
		\heiti \fontsize{45}{17} \quad \quad \quad 
		\vskip 1.5cm
		\heiti \zihao{2} 《计算科学导论》个人职业规划
	\end{center}
	\vskip 2.0cm
		
	\begin{quotation}
% 	\begin{center}
		\doublespacing
		
        \zihao{4}\par\setlength\parindent{7em}
		\quad 

		学生姓名:\underline{\qquad 鲜世洋 \qquad \qquad}

		学\hspace{0.61cm} 号:\underline{\qquad 1907020316\qquad}
		
		专业班级:\underline{\qquad 人工智能1901 \qquad  }
		
        学\hspace{0.61cm} 院:\underline{计算机科学与技术学院}
% 	\end{center}
		\vskip 1.5cm
		\centering
		\begin{table}[h]
            \centering 
            \zihao{4}
            \begin{tabular}{|c|c|c|c|c|c|c|c|c|}
            % 这里的rl 与表格对应可以看到,姓名是r,右对齐的;学号是l,左对齐的;若想居中,使用c关键字。
                \hline
                \multicolumn{5}{|c|}{分项评价} &\multicolumn{2}{c|}{整体评价}  & 总    分 & 评 阅 教 师\\
                \hline
                自我 & 环境 & 职业 & 实施 & 评估与 & 完整性 & 可行性 &\multirow{2}*{} &\multirow{2}*{}\\
                分析& 分析& 定位 & 方案 & 调整 & 20\% & 20\% & ~&~ \\\            
                10\% & 10\% & 15\% & 15\% & 10\% & &  &~ &~\\
                \cline{1-7} 
                & & & & & & & ~&~ \\
                & & & & & & & ~&~ \\
                \hline      
            \end{tabular}
        \end{table}
		\vskip 2cm
		\today
	\end{quotation}

\thispagestyle{empty}
\newpage
\setcounter{page}{1}
% 在这之前是封面,在这之后是正文
\section{自我分析}
	
\subsection{自然条件}
性别男,爱好女(说明性取向正),身体条件微胖,大体健康,比较容易感冒(因为在没有人监督的情况下,不太喜欢锻炼,所以抵抗力较低)。现居重庆,爱好广泛且正常,口味偏辣,眼睛度数左眼300度左右,右眼100度左右(不太适合长时间注视电脑),其他状况良好。\par
\subsection{性格分析}
讨厌一成不变的生活,喜欢与人打交道,有支配欲,喜欢影响和感染他人。在日常生活中与同学好友友好相处,待人热情,乐于助人,善于与别人建立亲密关系,行为大方慷慨,态度和蔼可亲,处事周密,得体,处理各种复杂人际关系游刃有余,对自己的行动、行为有责任感,受人尊重,受人欢迎。\par
待他人很热情,容易与人交往,外向。活泼,健谈,经常主动与他人交谈。在社交场合中较为轻松,与人交往表现得不卑不亢,但又不会过分突显。对多数人能较为公开展示自我,比较直率,但是不太喜欢在多人场合发言,并且会紧张。对他人没有有较强的依赖性,但是需要别人的支持才能维持自信心。\par
但是在做决策上面,思维敏捷,反应迅速,学习能力强。非常具有开放性,不太喜欢循规蹈矩,即关注事情的细节,又能从广阔的思路去考虑问题。处理事情容易感情用事,不能冷静思考分析。做事风格方面:做事时候,缺乏自信但很独立,遇到困难时喜欢自己解决,但这也会带来如同闭门造车的效应,不排斥与人共同工作。不排斥新事物和新观念,同时能够考虑到传统。倾向于接受外来强制标准和规则,但并不僵硬去遵从,经常能有效解决实际问题。能够克制自己,对事情能够进行事先计划和组织,有时候也会较为放任。\par
同时我情绪稳定,能够冷静应付现实,能振作勇气,维持团体的精神。但是不能短时间信任他人,需要长期相处交流才能比较信任他人。对自己的长处或缺陷有比较现实的认识,能为自己的失误承担责任。心平气和,容易紧张,对他人比较包容。大部分时间喜欢一个人呆着,但是和别人在一起,也不会有问题。\par
\subsection{教育与学习经历}
上小学时间较早,在小学自初中学习成绩还行,初中到高中阶段因为年少不懂事对学习不太感兴趣,所以学业成绩不太行。高一至高二阶段还是同初中一样,但是高三意识到问题的严重性,努力学习,但是因为高考时出现一部分失误再加上报考学校与专业时与父母没有沟通好,学校和专业不太满意,遂复读一年,然后来到中国石油大学(华东)计算机科学与技术学院软件工程专业然后报考本研一体班级专业转人工智能专业。\par
\subsection{工作与社会阅历}
工作与社会阅历,无工作经验,在初高中参加过几个比赛。对未来抱有希望和想象。\par
\subsection{知识、技能与经验}
仅有浅薄的基础知识,但是因为自己还是比较喜欢看书在书中得到的一些关于金融,计算机,数学的知识,这些知识没有深入的理解和运用。 \par 接受新鲜事物快,学习东西较快,会一点吉他,会一点乒乓球,有关生活方面的技能都还行,生活能够自理。\par 在初高中参加过几个比赛,遗憾的是奖项不太高。\par
\subsection{兴趣爱好与特长}
电脑游戏有些研究(仅限于游戏的玩法,和游戏的种类),爱好音乐,爱好乒乓球,兴趣还是偏文一些(比如对金融,对各种思想,对名人传记感兴趣),但是对于工科方面具体到生活实际方面的问题非常感兴趣(比如研究智慧医疗,安全系统等),对于理论问题不感兴趣。\par
\section{环境分析}

\subsection{社会环境分析}
在中国共产党的带领下,中国政治形势一如既往的良好,但国际形势不容乐观,如香港问题,以美国为首的资本主义帝国亡我中华之心不死,对我国开展贸易战等限制我国发展的手段,对我国经济是肯定会产生影响的,但贸易全球化是不可挡的。\par 单边主义,美国不顾其他国家反对,为自己利益采取损害其他国家利益,是肯定会失败的,可这些不良因素势必在短期甚至中期对国民经济产生冲突,这也要求我们以国家利益为主,好好学习。\par  国内关于高科技的就业形势还是经济形势不仅没有收到影响,反而稳中有升,说明国家因为科技的封锁更加重视科研了,和大学教育了,对我们更有好处。\par 总而言之,这里引用习总书记的一句话当今世界正经历百年未有之大变局,但和平、发展、合作、共赢的时代潮流没有变。对于大学生是个机遇也是个挑战,未有强大自身才能成为弄潮儿。\par
\subsection{家庭环境分析}
现阶段无配偶(甚至连女朋友也没有,希望在不久的将来有),经济没有独立,父母给的生活费能够满足花销,家里和睦,父母健康,有稳定的收入,对我期望一般(但自我期望较高),但是对于我身体,道德,做人方面要求高。家族对于学习才是唯一的出路认识很深,因此对大家学业要求高,希望其能够成才。\par
\subsection{职业环境分析}
\begin{itemize}
    \item 对于大环境下的IT行业形势\par
中国IT产业主要包括电子信息产品的制造、软件开发、信息技术服务的推广应用等。经过改革开放和快速发展,目前我国的信息产业已形成了较为完整的工业生产体系。全国电子工业总产值规模已居世界第四位,主要电子产品已形成规模化生产,其中收录音机、电话机、彩电、彩管、音响设备、VCD和一些基础元器件的生产规模已居世界第一位。但是相比较欧美IT企业以及后起之秀的日本、韩国,中国IT制造业还处于产业的下游。中国许多从事加工、装配的IT企业深受价格战、高额专利费等问题的困扰。同时,随着世界上其他不发达地域的开发,中国IT制造业原来具有的劳动力和资源便宜优势也面临着愈来愈激烈的竞争。\par 纵观世界经济的发展,经济全球化进程明显加快,信息化已成为全球化的迫切需要和必要保证。世界范围的产业结构调整和信息技术进步,必将对中国信息产业的发展产生深刻影响。\par
众所周知,信息产业是国民经济的主导产业,是经济增长的催化剂和倍增器。根据国务院批准的“三定方案”,信息产业部的主要任务是:通过积极有效的宏观管理,振兴电子信息产品制造业、软件业和通信运营业,为各部门、各行业提供先进的信息技术、装备与网络服务,从而达到推进国民经济发展和社会服务信息化的目的。因此无论是我国对于IT行业的重视还是中国乃至世界对于IT行业对需求,都说明这个行业正处于还会继续处于上升的阶段。
\item 对于个人来说IT行业形势 \par IT行业是最公平的,没人靠关系、靠背景,你的能力决定一切,而且从事 IT 行业之后你的视野会更开阔,你可以接触以前从来没接触的东西,你可以随时关注到这个世界的变化,你的格局、思维、想法都会发生改变,并且收入可观,能够很好的通过自己晋升岗位。无论是现在,还是未来发展,在选择的时候,完全可以将IT行业视为常青的职业。
\par
\end{itemize}

\subsection{地域与人际环境分析}
重庆IT行业潜力还是挺大的,但是与毗邻的成都还有三五年的差距,虽然成都比北上广深还差远了。重庆政府相关部门一直还是希望把互联网行业做起来,但是经历西永电子产业园流产,到两江新区试图再次重点打造软件园区和建立互联网学院来培养当地软件人员,都没形成气候。现在表现出来的就是,没有一个成型的集中的互联网产业园区,北京有中关村,成都有高新区软件园,而重庆有很多软件园却又没有一个真正的软件园。企业上,除了本地的猪八戒网还有点名气,很少有叫得上名号的企业了,更多的就是一些外包公司入驻。而且在重庆IT行业普遍薪资较少,但是因为从小出生,学习都在重庆,因此在重庆工作对于我还是很好的,唯一的缺憾是大学是在青岛与重庆的关联性不大,但是对我工作而言也是多一个选择。 \par
\par 




\section{职业定位}


\subsection{行业领域定位与理由}
未来我想从事有关于AI在生活中的运用,理由如下\par
\begin{itemize}
    \item 第一是因为我喜欢看见我所从事的职业对人类有用。
    \item 第二是因为这是我现在所学的内容。
    \item 第三是因为AI也是国家发展的战略,世界急需相关人才。\par
\end{itemize}
其次是关于网络安全相关的工作。\par
\subsection{职业岗位起点定位与理由}
根据自己的职业兴趣与个人能力,我充分的认识自己的优缺点,我需要从行业的最底层做起,充分的认识到这个行业的底层架构和未来发展方向,逐步成长为一名AI偏向与管理人才,因此我将自己的职业岗位起点定位于基层码农。\par
\subsection{职业目标与可行性分析}
\par
不断的学习,稳中提高,不急不燥。\par 
\begin{enumerate}[(1)]
	\item 短期目标(大学4年)\par 
	在大学4年中首先完成好课堂内容,加强身体锻炼,最好谈 一次恋爱,然后希望加入到导师的课题组,提前感受一下科研的氛围,争取在大学4年中获得一些奖项或者完成一些对人类有意义的课题。\par
	\item 中长期目标(5-10年)。\par
	在社会中立足,身体健康,有幸福的家庭,有稳定且可观的收入,完成了一个或者多个对人类有意义对整个IT行业有影响的项目,努力成为业界大牛。\par
\end{enumerate}



\section{实施方案}
\begin{itemize}
    \item 讨论交流方法。\par
   \begin{enumerate}[1、]
	\item 在校期间多和老师讨论交流,毕业后选择后其中有见解的人保持交流。
	\item 在工作中多和直接上司沟通交流,加深了解,建立一个良好的印象。

\end{enumerate}
    \item 实践锻炼方法。\par
  
\begin{enumerate}[1、]
	\item 锻炼自己的注意力,在嘈杂的环境中也能思考,正常工作。
	\item 养成良好的饮食,锻炼,生活习惯,才能有本钱好好工作。
	\item 利用自身的优势,扩大交友圈,重视朋友的交际圈。
	\item 多参加一些对自己有磨练的竞赛或者课题,提前适应科研生活。
\end{enumerate}
\end{itemize}
\par 

\section{评估与调整}
世间万事总在不断变化发展,因此我要对自己的职业生涯规划作出一些风险预测,并制定相应的调整解决方案。
\par 
\subsection{评估时间}
每学年一次。\par
\subsection{评估内容}
\begin{itemize}
    \item 成果目标评估\par
    如果在大学4年未完成一次竞赛或者参加一次课题,即视为失败。
    \item 经济目标评估\par
    在大学毕业后2年争取经济独立,并在4年内保持稳定且较高的薪资。
    \item 能力目标评估\par
    与成果目标一同评估。
     \item 职务目标评估\par
     在刚毕业的时期不争取管理岗位,努力学习技术,在毕业后4年内力争成为公司中上层。
\end{itemize}\par
\subsection{调整原则}
\begin{itemize}
    \item 清晰性原则\par
    如果在工作或学习中对未来规划不清晰,应该给与适当时间进行调整。
    \item  变动性原则\par
    能够根据行业大环境变动而改变。
    \item  挑战性原则\par
    如果这个行业不再具有挑战性,可以适当考虑改变行业。
     \item  激励性原则\par
     选择的行业应当对我具有激励作用。
\end{itemize}
\section{结束语}
非常感谢老师能够给与我们这个机会去看清自己在大学乃至人生以后的目标,并且帮助我们规划。职业生涯规划是一个有机、持续不断的探索过程,随着自身条件和外部环境的变化而变化,但是不管外环境怎么变化,成为一位优秀的IT从业人员都是自己成长和学校培养我们的目标,我同时也希望在未来能够在这一行业发光发热。\par


\end{document}
